\documentclass[UTF8,a4]{article}

\usepackage{CTEX,amsmath,amssymb}
\usepackage{geometry}

\geometry{left=3cm,right=2.5cm,top=2.5cm,bottom=2.5cm}

\begin{document}

\title{Generalized Benders Decomposition 学习笔记}
\maketitle

\section{原始问题}

原始问题如下:
\begin{equation}
\label{original}
\begin{split}
&\max_{x,y}f(x,y)\\
&s.t.\quad G(x,y)\succeq 0,x\in X,y\in Y\\
\end{split}
\end{equation}
其中,$G(x,y)$所得结果为一个$m$维向量(也即有m个不等式约束),假设在\eqref{original}中,y是复杂变量,即当y是固定值的时候,\eqref{original}就会变得很好解,这主要体现在一下几个可能的方面:
\begin{itemize}
\item[(a)]对于固定的y,问题\eqref{original}可以分解为若干个独立的优化问题,每一个子优化问题都是x的一个子向量,且互相不交叠;
\item[(b)]问题\eqref{original}可能是一个可以有效求解的著名的问题结构;
\item[(c)]问题\eqref{original}对于x和y联合起来并不是concave的,但是固定y之后对于x就是concave的。
\end{itemize}
为了将y固定,进行的操作是将问题投影到y空间上,(projection/partitioning),经过映射或者是投影之后,问题变为了如下的形式:
\begin{equation}
\label{2}
\begin{split}
&\max_y \quad v(y)\\
&s.t.\quad y\in Y\cap V
\end{split}
\end{equation}
其中,$v(y)$与$V$的定义如下:
\begin{equation}
\label{3}
\begin{split}
v(y):=\sup_x f(x,y)\\
s.t. \quad G(x,y)\succeq 0,x\in X
\end{split}
\end{equation}
\begin{equation}
\label{4}
V:=\{y|G(x,y)\succeq 0,\exists x\in X\}
\end{equation}
实际上,Y是y原来的取值范围,而V则是可行的y的取值范围,两者求交集,则是在限定取值范围内的可行解y的范围。
注意到,$v(y)$是在给定$y$的情况下,\eqref{original}的最优取值,因为\eqref{3}太常用了,所以用(1-y)来指代\eqref{3}中的优化问题。
$$\max_{x\in X}f(x,y),s.t. G(x,y)\succeq 0 \quad\quad\quad\quad(1-y)$$
在Benders原始的工作中,只考虑了通过线性对偶理论来定义目标函数和限制条件,这样v和V就可以用有限数量的近似就能得到有限数量的近似。
\begin{subequations}
\begin{align}
X&:=\{x|x\succeq0 \}\\
f(x,y)&:=c^tx+\phi(y)\\
G(x,y)&:=Ax+g(y)-b
\end{align}
\end{subequations}
但是本文的主要工作是对Benders工作的扩展,扩展到非线性对偶理论的一般情况下,这样可以解决的问题的范围就增加扩大了。
\section{泛化的Benders分解}
这一章主要从5个方面来讨论\eqref{original}这样的问题的Benders分解。第1部分讨论的是如何推导出master问题,核心的思想就在变为\eqref{2}之后利用对偶的形式来标识v和V;第2部分主要是通过解决一系列的子问题subproblem来解决master问题(主要是通过计算(1-y)不同试验y值的最优乘子向量);第3部分主要陈述了benders分解的主要步骤;第4部分讨论了收敛性的问题,第5部分讨论了计算上的一些问题。
\subsection{Master Problem的推导}
master problem的获得主要通过三步来达到:
\begin{itemize}
\item[(i)]将\eqref{original}映射到y空间,得到\eqref{2}的形式
\item[(ii)]触发V的natural dual representation(自然对偶表示?),也就是一些包含V区域的重合的部分,invoke the natural representation of V in terms of the intersection of a collection of regions  that contain it
\item[(iii)]触发v的natural dual representation(自然对偶表示?),invoke the natural dual representation of v in terms of the pointwise infimum of a collection of functions that dominate it.
\end{itemize}
\subsubsection{步骤1,Projection}
\textbf{\emph{定理1,Projection}}
当且仅当问题\eqref{2}不可行或者是无上界时,问题\eqref{original}才会不可行或者是无上界。如果$x^*,y^*$是\eqref{original}的最优解,那么$y^*$一定是\eqref{2}的最优解。如果$y^*$是\eqref{2}的最优解,并且在$y=y^*$时通过$x^*$达到了\eqref{3}的最大值,则$(x^*,y^*)$同样也是\eqref{original}的最优解。如果说$\bar{y}$是\eqref{2}的$\epsilon_1-optimal$,$\bar{x}$是(1-y)的$\epsilon_2-optimal$,则$(\bar{x},\bar{y})$是问题\eqref{original}的$(\epsilon_1+\epsilon_2)-optimal$
\subsubsection{步骤2,V-Representation}
\textbf{\emph{定理2,V-Representation}}
假设1:X是非空凸集,对于给定的$y\in Y$,$G$在X上是concave的;假设2:对于给定$y\in Y$,集合$Z_y:=\{z\in R^m|G(x,y)\succeq z, \exists x\in X\}$是闭集(这一假设实际上假设的是对于给定的y,G(x,y)的每一个维度的最大值都可以取到,而不是开集,最大值也不是无穷的(?))。那么,对于点$\bar{y}\in Y$,当且仅当\eqref{6}成立时,其也在集合V中。
\begin{equation}
\label{6}
\begin{split}
&[\sup_{x\in X}\lambda^tG(x,\bar{y})]\succeq 0,\forall\lambda\in\Lambda\\
where\\
&\Lambda:=\{\lambda\in R^m|\lambda\succeq 0\quad and \quad \sum_{i=1}^m\lambda_i=1\}
\end{split}
\end{equation}
\textbf{\emph{证明}},如果点$\bar{y}$在V中,\eqref{6}自然是成立的,也不必证明。反过来,通过\eqref{6}来证明$\bar{y}$则可以通过非线性对偶理论。假设$\bar{y}$满足\eqref{6},那么
$$
\inf_{\lambda\in\Lambda}[\sup_{x \in X} \lambda^t G(x,\bar{y})]\geqslant0
$$
进一步推导,得出
\begin{equation}
\label{7}
\inf_{\lambda\geqslant0}[\sup_{x\in X}\lambda^tG(x,\bar{y})]=0
\end{equation}
通过\eqref{7},可以断言以下concave优化\eqref{8} 关于$G$约束的对偶函数($inf_{\lambda\geqslant0}[\sup_{x\in X}\lambda^tG(x,\bar{y})]$)是可以达到最优值0的(即\eqref{8}的对偶是有可行解的,对于给定$\bar{y}$,存在x使得$G(x,\bar{y})\geqslant 0$)。
\begin{equation}
\label{8}
\begin{split}
\max_{x\in X}&\quad 0^t x\\
s.t.&\quad G(x,\bar{y})\geqslant0\\
\end{split}
\end{equation}
根据参考定理5.1,如果\eqref{8}的对偶是有可行解的,那么原问题\eqref{8}也必定有可行解,因此,可以证明$\bar{y}\in V$。另外,关于$Z_y$是闭集的限制也不是严格的,有些情况可以不必是闭集。
\subsubsection{步骤3, v-Representation}
\textbf{\emph{定理3}}假设1:X是非空凸集,$f$和$G$在X上对于给定的$y\in Y$是concave的;假设2:对于给定的$\bar{y}\in Y\cap V$,以下三种情况至少有一种会出现:
\begin{itemize}
\item[(1)]$v(\bar{y})$有限,且(1-$\bar{y}$)拥有\textbf{最佳乘子向量}(定义见附录,大意是在拉格朗日对偶的情况下,最佳乘子向量与最优解条件下的限制条件满足互补松弛性);该情况是常见的情况
\item[(2)]$v(\bar{y})$有限,$G(x,\bar{y})$与$f(x,\bar{y})$在X上连续(X是闭集),同时$\exists \epsilon\geqslant 0$使得(1-$\bar{y}$)问题的$\epsilon-optimal$解集是非空并且有限的;
\item[(3)]$v(\bar{y})=+\infty$
\end{itemize}
如此一来,(1-y)的最优值等于它的对偶函数在$Y\cap V$上的最优取值,即
\begin{equation}
\label{9}
\begin{split}
v(y)&=\inf_{u\succeq0}[\sup_{x\in X}f(x,y)+u^tG(x,y)]\\
&\forall y\in Y\cap U
\end{split}
\end{equation}
\subsubsection{总结}
经过上述三个步骤的定理与证明,问题\eqref{original}可以被转变为如下等价的master problem形式:
$$
\max_{y\in Y}[\inf_{u\succeq0}[\sup_{x\in X}f(x,y)+u^tG(x,y)]],\quad s.t. \eqref{6}
$$
或者,另一种表述形式是利用infimum是最小下界的定义,将上述问题转化
为如下的形式:

\begin{subequations}
\label{10}
\begin{align}
\label{10a}
&\max_{y\in Y,y_0}y_0\\
\label{10b}
s.t.\quad &y_0\leqslant\sup_{x\in X}\{f(x,y)+u^tG(x,y)\},\forall u\succeq0\\
\label{10c}
&\sup_{x\in X}{\lambda^tG(x,y)}\geqslant0, \forall\lambda\in\Lambda
\end{align}
\end{subequations}
其中,\eqref{10a}是将上式中的$\max_{y}(\cdot)$换成了一个变量,然后\eqref{10b}是将inf的定义转化为最小下界,即令$\sup_{x\in X}\{f(x,y)+u^tG(x,y)\}$取得最小值的u带入之后所得的最小值,与inf一致,最后\eqref{10c}则是保证\eqref{6}的限制被满足,即$y\in V$。
\subsection{求解master problem}

因为master problem一般情况下是有大量的限制条件的,因此为了提升求解的效率,直接的做法是进行松弛(relaxation)。最开始首先忽略大部分的约束条件,只考虑\eqref{10b}\eqref{10c}中的部分限制条件,进行求解,将所求的“最优解”(次出的最优解既可以指准确的全局最优,也可以指$\epsilon-optimal$)代入到先前被忽略的限制条件中去检验是否满足所有限制,如果满足,则达到了最优解;如果不满足,则加入被违背的限制条件,重新求解,直到求得满足所有限制条件的最优解。
上述思想比较简单,但是在这个过程中存在着两个问题,第一个问题是针对松弛问题所求得的最优解,如何去验证它们满足所有限制条件;第二个是如果是不满足所有限制条件的话,哪些限制条件需要被加到新的松弛优化问题。
解决上述问题的一个思路如下:假设$(\hat{y},\hat{y_0})$是松弛问题\eqref{10}的最优解,那么我们需要验证其满足\eqref{10b}\eqref{10c}的限制。从定理2与V的定义来看,当且仅当(1-$\hat{y}$)有可行解的时候,$\hat{y}$才能满足\eqref{10c}的限制要求(因为(1-$\hat{y}$)有可行解的限制条件就是对于指定的$\hat{y}$,可行解满足$G(x,\hat{y}\succeq 0)$,自然有$\sup_xG(x,\hat{y})\succeq 0$)。另外,如果(1-$\hat{y}$)有可行解,定理3表明当且仅当松弛最优解小于等于考虑了所有限制条件的最优解即$\hat{y_0}\leqslant v(\hat{y})$时,$(\hat{y},\hat{y_0})$满足约束\eqref{10b}的要求(定理3指明在给定的$\hat{y}$的情况下,考虑了所有约束条件的(1-$\hat{y}$)的最优取值为$v(\hat{y})$,如果$\hat{y_0}\geqslant v(\hat{y})$的话,则说明其一定违背了某些限制条件)。如此一来,(1-$\hat{y}$)便成了验证松弛最优解是否可行的非常合适的子问题subproblem,而在给定松弛最优解的自变量$\hat{y}$时,(1-$\hat{y}$)就变得非常容易求解。
假设松弛最优解$(\hat{y},\hat{y_0})$违背了被忽略的限制条件,在求解(1-$\hat{y}$)的过程中,如果(1-$\hat{y}$)没有可行解,则大多数的对偶类型的解法都会计算出向量$\hat{\lambda}$使得\eqref{11b}被满足;如果(1-$\hat{y}$)有可行解,且能计算出一个有限的最优解,则会算出一个向量$\hat{u}$满足\eqref{11a},实际上大多数现在的解法都能解出$\hat{u}$,如果解不出来,则一定是有无界的最优,对应着原问题\eqref{original}也是有着无界最优,或者是有有限最优解,但是处于一种异常状态,使得通过\eqref{9}算出的$\hat{u}$当其足够靠近最优的时候满足\eqref{11a},这样的$\hat{u}$被称为near-optimal乘子向量。
\begin{subequations}
\begin{align}
\label{11a}
\hat{y_0}>&\sup_{x\in X}\{f(x,\hat{y})+\hat{u}^tG(x,\hat{y})\}\\
\label{11b}
&\sup_{x\in X}\{\hat{\lambda}^tG(x,\hat{y})\}<0
\end{align}
\end{subequations}
总之,通过对偶的方式求解(1-$\hat{y}$)可以有效的验证松弛最优解是否合理:如果(1-$\hat{y}$)不可行,则会算出$\hat{\lambda}\in\Lambda$满足违背条件\eqref{11b};如果可行,可以算出最佳乘子向量$\hat{u}$满足\eqref{11a};或者算不出最佳乘子向量,但是可以算出near-optimal的乘子向量满足\eqref{11a}($\hat{y_0}$超过了$v(\hat{y})$)。
这里可能会有个疑问,为什么\eqref{10b}里边的限制条件明明限制了$y_0$的取值不能大于$v(y)$,为什么还会算出$\hat{y_0}>v(\hat{y})$呢?这是因为在实际的求解master problem过程中,并没有考虑完整的限制条件,而是只是考虑了部分的限制条件,所以才可能出现算出的$\hat{y_0}$大于考虑所有限制条件的全局最优解$v(\hat{y})$
\subsection{求解过程描述}
假设:定理2与定理3的假设依旧成立,同时假设问题\eqref{original}有有限的最优解。
定义函数如下:
\begin{subequations}
\begin{align}
\label{12a}
&L^*(y;u):=\sup_{x\in X}\{f(x,y)+u^tG(x,y)\},&y\in Y,u\succeq 0\\
\label{12b}
&L_*(y;\lambda):=\sup_{x\in X}\{\lambda^tG(x,y)\},&y\in Y,\lambda\succeq0
\end{align}
\end{subequations}
具体步骤如下:
\begin{itemize}
\item[Step 1]选择$\bar{y}\in Y\cap U $,解决子问题(1-$\bar{y}$),获得最佳或者near-optimal乘子向量$\bar{u}$,获得函数$L^*(y;\bar{u})$。令$p=1,q=0,u^1=\bar{u},LBD=v(\bar{y})$,选择收敛容忍度参数$\epsilon>0$。其中,LBD是可行解能取得的最优值的下界,即全局最优值至少能达到LBD。
\item[Step 2]通过适当的方法求解当前迭代阶段的松弛主问题\eqref{13},所得到的解$(\hat{y},\hat{y_0})$为当前松弛最优解,$\hat{y_0}$为问题\eqref{original}的最优值的上界(即考虑了部分的限制条件下的最优解,拥有更多限制条件的原始问题不可能取得比$\hat{y_0}$更好的解了),如果$LBD\geqslant\hat{y_0}-\epsilon$,则迭代停止。
\begin{equation}
\label{13}
\begin{split}
\max_{y\in Y,y_0}\quad&y_0\\
s.t.\quad&y_0\leqslant L^*(y;u^j), j=1,2,\cdots,p,\\
&L_*(y;\lambda^j)\geqslant 0, j=1,2,\cdots,q,\\
\end{split}
\end{equation}
\item[Step 3]求解更新过的子问题(1-$\hat{y}$),然后检查下列情形:
\item[Step 3A]\emph{$v(\hat{y})$的值是有限的},如果$v(\hat{y}) \geqslant \hat{y_0} - \epsilon$,则迭代终止,达到了$\epsilon-optimal$。否则,说明当前的松弛最优解$\hat{y}$导致部分限制条件被违背,所以原问题的最优解\eqref{original}无法通过$\hat{y}$来达到合理的程度,通过再次求解(1-$\hat{y}$)获得新的最优乘子向量$\hat{u}$(如果求不到,则找near-optimal)进而得到新的函数$L^*(y;\hat{u})$。将p加1,令$u^p=\hat{u}$,如果$v(\hat{y})>LBD$,则令$LBD=v(\hat{y})$,更新LBD,($v(\hat{y})$必定是可达到的),然后重复step 2进行检查。
\item[Step 3B]\emph{问题(1-$\hat{y}$)不可行},则算出满足\eqref{11b}的$\hat{\lambda}\in\Lambda$,同时得到函数$L_*(y;\hat{\lambda})$,令q加1,然后令$\lambda^q=\hat{\lambda}$,然后返回Step 2.
\end{itemize}

有几点值得注意。(1)在Step2中,连续找到的$\hat{y_0}$必须是单调非增的,而在Step3中,找到的$v(\hat{y})$则不必也未必是单调非减的,这也是引入LBD的原因。(2)令人欣慰的是,可以证明,在Step3A中,所得到的最优乘子向量$\hat{u}$所对应的约束条件也是被违背最多的约束条件(直观来看,$\hat{u}$应该是会在$G(x,\hat{y})$为负的维度上数值最大,以便实现$\hat{u}^tG(x,\hat{y})$的最小),而near-optimal乘子向量能够在多大程度上刻画限制条件的被违背程度,则取决于其能够在多大程度上接近(1-$\hat{y}$)对偶的最优。同理,$\hat{\lambda}$则取决于当$\bar{y}=\hat{y}$时其多大程度达到\eqref{8}对偶的最优(对偶变量正规化到$\Lambda$上)。

\subsection{收敛性分析}

\subsection{计算细节}

\end{document}
