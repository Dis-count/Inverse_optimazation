\documentclass[UTF8]{article}

\usepackage{amsmath}
\usepackage{graphicx}
\usepackage{setspace}
% \usepackage{ctex}
\usepackage{geometry}
\newtheorem{thm}{Theorem}
\newtheorem{pf}{Proof}
\newtheorem{algorithm}{Algorithm}
\numberwithin{equation}{section}
% \geometry{a4paper,left=2cm,right=2cm,top=1cm,bottom=1cm}
% \setstretch{1.5}
\geometry{a4paper,scale=0.8}
\renewcommand{\baselinestretch}{1.5}

\title{Inverse optimization}
% \author{Dis \cdot count}
\date{Dec 2019}
\begin{document}
\maketitle{}

\section{The Model}

The original LP model is:

\begin{equation}
\begin{align*}
  \min \quad c^Tx \\
\text{s.t.} \quad A'x \geq b  \\
x \geq 0
\end{align*}
\end{equation}

The dual of the LP is:
\begin{equation}
\begin{align*}
  \max \quad b^Ty \\
\text{s.t.} \quad A^{'T}y \leq c  \\
y \geq 0
\end{align*}
\end{equation}

Among this, $y$ is dual variable.
Thus, the inverse optimization model can be expressed as the following:

\begin{align}
  \min \Vert A'-A \Vert_1 \nonumber \\
\mathrm{s.t.}  \quad A^{'T} \pi \leq c  \\
  b^T \pi = c^Tx^0  \\
  \quad A'x^0 \geq b \label{Sol_con}
\end{align}

We want to obtain $(\pi,A')$, and $(b,c,x^0)$ is given.

We can split this into two situations: given the value or given the solution. If we give the solution, just as the model we showed. If we give the value, we don't need the (\ref{Sol_con}) constraints.
Expand the matrix to the specific elements. We can obtain the corresponding model:

\begin{equation}
\begin{align*}
& \min \quad \sum_i \sum_j (e_{ij}+f_{ij})\\
\mathrm{s.t.} \quad &\sum_{i=1}^m a^{'}_{ij} \pi_i \leq c_j \\
&\sum_{j=1}^n a^{'}_{ij} x_j^0 \geq b_i \\
&\sum_{i=1}^m b_i \pi_i = v_0 \\
% e_{ij}f_{ij} = 0   \\
&a^{'}_{ij} = e_{ij}-f{ij}+a_{ij} \\
&e_{ij} \geq 0 \qquad f_{ij} \geq 0
\end{align*}
\end{equation}

Notice that there are $(3mn+m)$ variables but only $(mn+m+n+1)$ constraints.

The theoretical method is to use the KKT conditions which is usually used to solve the constrained nonlinear programming.
Add the lagrangian multiplier:

\begin{align}
  T(e,f,y) &= \min \sum_i\sum_j(e_{ij}+f_{ij}) + \sum_{j=1}^n \lambda_j
  g_j(e,f,y) + \sum_{i=1}^m \mu_i f_i(e,f,y)\nonumber\\
\mathrm{s.t.}  \quad   g_j(e,f,y) &= \sum_{i=1}^m (e_{ij}-f_{ij}+a_{ij})y_i-c_j = 0 \tag{\text{God}} \\
  f_i(e,f,y) &= b_i - \sum_{j=1}^n (e_{ij}-f_{ij}+a_{ij})x_j^0 \leq 0 \\
  h(y) &= \sum_{i=1}^m b_i y_i - v_0 \leq 0\\
  K(e,f) &= e_{ij}f_{ij}=0  \\
  M(e) &= -e_{ij} \leq 0 \label{M}  \\
  N(f) &= -f_{ij} \leq 0
\end{align}

The corresponding multipliers are $\lambda_j, \mu_i, \alpha, \beta_{ij}, m_{ij}, n_{ij}$.

Using the KKT constraints, we can obtain the following equations.

\begin{align*}
  \frac{\partial T(e,f,y)}{\partial e_{ij}} &= 1+ \beta_{ij} f_{ij} - m_{ij}+\lambda_j y_i + \mu_i(-x_j^0) = 0  \\
  \frac{\partial T(e,f,y)}{\partial f_{ij}} &= 1+ \beta_{ij} e_{ij} - n_{ij}-\lambda_j y_i + \mu_i x_j^0 = 0   \\
  \frac{\partial T(e,f,y)}{\partial y_{i}} &=\sum_{j=1}^n(e_{ij}-f_{ij}+a_{ij})\lambda_j + \alpha b_i = 0 \\
  g_j(e,f,y) &=0 \\
  \mu_i f_i &= 0 \quad \mu_i \geq 0 \\
  K(e,f) &= e_{ij}f_{ij} =0 \\
  \alpha h(y) &=0 \quad \alpha \geq 0\\
  m_{ij} M(e) &=0 \quad m_{ij} \geq 0 \\
  n_{ij} N(f) &=0 \quad n_{ij} \geq 0
\end{align*}

If we can solve these equations, we can obtain the solution $(e_{ij}^*, f_{ij}^*, y_i^*)$. And the $e_{ij}^*, f_{ij}^*$ corresponds the optimal adjustment, the $y_i^*$ corresponds the optimal lagrangian multipiers.
However, the system of nonlinear equations doesnot seem to be solved in this case.

\section{Analysis}

Consider the particularity of this problem.


% 跟师兄聊天的一点启发
% 一是文章的套路问题

% 二是对于这个问题可以有下面几个方面改进

% min 可以不用KKT嘛   还是根据性质进行优化 但如果还是 非线性约束优化问题还是要用KKT

% min 通过 KKT 得到 一组非线性方程组 没有理论解  但可以先根据性质分析不同情形 再不行就找到不等式放缩 找到解的大致位置 再退而求其次  利用性质结合已有的方法 进行优化 得到迭代解

% 可能存在的几种情况
% 最优 f_i = 0 , h(y) = 0
% 其中 若令所有 f_i = 0 则会得到一个上界 而满足 f_i 的所有不等式 均为改变后的 active 不等式

% 还有一种思路是 能不能把一部分非线性解决 而另一部分线性不管它

Case 1. All the $f_{ij} = 0, e_{ij} > 0$

The original model is transformed as:

\begin{equation}
\begin{align*}
&G =  \min \quad \sum_i \sum_j e_{ij} \\
\mathrm{s.t.}  \quad &\sum_{i=1}^m (e_{ij} + a{ij})y_i & = c_j \\
&\sum_{j=1}^n (e_{ij} + a{ij})x_j^0 & \geq b_i \\
&\sum_{i=1}^m b_i y_i & \leq v_0 \\
% e_{ij}f_{ij} = 0   \\
&e_{ij} \geq 0
\end{align*}
\end{equation}

If all $a_{ij} > 0$, we can further transform this model as a geometric programming(GP), because the functions are all posynomialfunction.
Let $z_i = \ln x_i \Rightarrow x_i = e^{z_i}$, and $e_{ij} = x_k,~ k=1,\ldots,i\times j$, $y_i = x_k, k=i\times j+1,\ldots,i\times j+m$

$f(x)=\sum_{k=1}^{K} c_{k} x_{1}^{a_{1 k}} x_{2}^{a_{2 k}} \cdots x_{n}^{a_{n k}}=\sum_{k=1}^{K} e^{a_{k}^{T} y+b_{k}}$

% 第二个等式不是 一般形式 如何求解 需要更改

Then, the model can be written as:
% 有2m项加和 n项 等式
\begin{equation}
\begin{align}
&\min \quad \sum_k^{i\times j} e^{y_k} \\
\mathrm{s.t.}& \quad \sum_{k=1}^{2m} e^{a_{ik}^{T} y+b_{ik}} = 1, i =1,\ldots,n \\
&\sum_{k=1}^{2m} e^{a_{ik}^{T} y+b_{ik}} \leq 1, i =1,\ldots,n \\
\end{align}
\end{equation}

Because the optimal solution satisfies the

And the system is changed as: $h(y)=0$ and

\begin{align*}
  1 - m_{ij}+\lambda_j y_i + \mu_i(-x_j^0) = 0  \\
  1+ \beta_{ij} e_{ij} - n_{ij}-\lambda_j y_i + \mu_i x_j^0 = 0   \\
  \sum_{j=1}^n(e_{ij}+a_{ij})\lambda_j + \alpha b_i = 0 \\
  \sum_{i=1}^m (e_{ij}+a_{ij})y_i-c_j = 0 \\
  b_i - \sum_{j=1}^n (e_{ij}+a_{ij})x_j^0 \leq 0 \\
  \mu_i f_i = 0 \quad \mu_i \geq 0 \\
  \sum_{i=1}^m b_i y_i = v_0 \quad \alpha \geq 0 \\
  m_{ij} e_{ij} =0 \quad m_{ij} \geq 0 \\
  n_{ij} \geq 0
\end{align*}

Let all $m_{ij} = 0$

Another information is that if $x_j^0$ satisfies some equation
$f_i = 0, i \in I$. For other $i \in \{1,\dots,n\}\backslash I, f_i\leq 0 ~\text{and}~ \mu_i = 0$.


\end{document}
