\documentclass[UTF8]{article}
\author {Dis\cdot count}

\title {设施选址逆优化问题的计算}
\date{}
\usepackage{ctex}
\usepackage{amsmath}

\usepackage{geometry}
\geometry{a4paper,scale=0.8}
\usepackage{graphicx}
\usepackage{amssymb}

\usepackage{setspace}
\renewcommand{\baselinestretch}{1.5}


\usepackage{float}
\usepackage{color}%,soul}f
\usepackage{multirow}
\usepackage{xr}


\begin{document}
    \maketitle

\begin{abstract}

逆优化问题

我们使用了经典的列生成算法对UFL的逆优化问题进行了计算,其子问题就是求解原问题UFL,这样求解得到的就是UFL的逆优化最优值。当对子问题进行优化时,例如对子问题进行松弛得到非整数解这样可以得到与启发式方法一样的上界,而对子问题得到的非整数解取临近整数可以得逆问题的下界


\end{abstract}

\qquad \textbf{关键词: 逆优化、列生成算法、启发式算法 、设施选址问题}

\section{逆优化综述}  % Introduction

逆优化问题最初
已经有很多人进行了研究
同时设施选址
由于设施选址问题本身是NP难问题,它的逆问题就变得更为复杂。若是

设施选址逆优化

\section{无容量限制}

设施选址问题通常被提出如下:假设有$m$个设施和$n$个客户。我们希望选择(1)哪些设施要打开,以及(2)哪些打开的设施要用于向哪些客户提供,以便以最低的成本满足某些固定需求。引入以下记号,令$f_i$ 表示开启设施$i$的固定成本,$i \in M, M=\{1,\ldots,m\}$。$c_{ij}$ 表示运输商品从设施$i$到客户$j$的成本,$j \in N, N=\{1,\ldots,n\}$。更进一步,令$d_j$ 表示客户$j$的需求,并且假设每一个设施有一个最大输出限制,令$k_i$表示设施$i$能够生产出的最大商品量,即表示设施$i$的容量。

则无容量限制设施选址问题定义如下:
\begin{align*}
&\min \sum_{i=1}^n \sum_{j=1}^m c_{ij}u_{ij} + \sum_{i=1}^n f_i v_i \\
\text{s.t.}& \sum_{i=1}^n u_{ij} =1, \forall j \in \{1,\ldots,m\}  \\
&\sum_{j=1}^m u_{ij}  \leq Mv_i, \forall i \in \{1,\ldots,n\} \\
& u_{ij} \in \{0,1\}, \quad v_{i} \in \{0,1\}
\end{align*}

它的逆优化问题定义如下:



类似的,有容量限制的设施选址问题定义如下:
\begin{align*}
&\min \sum_{i=1}^n \sum_{j=1}^m c_{ij}u_{ij} + \sum_{i=1}^n f_i v_i \\
\text{s.t.}& \sum_{i=1}^n u_{ij} =1, \forall j \in \{1,\ldots,m\}  \\
&\sum_{j=1}^m d_j u_{ij}  \leq k_iv_i, \forall i \in \{1,\ldots,n\} \\
& 0 \leq u_{ij} \leq 1,\forall j \in \{1,\ldots,m\}, \forall i \in \{1,\ldots,n\} \\
& v_{i} \in \{0,1\}, \forall i \in \{1,\ldots,n\}
\end{align*}


\subsection{列生成算法得到最优解}
列生成算法作为一种有效求解规模较大的线性规划问题。
原逆优化模型为
\begin{align*}
&\min |c-c_0|  \\
\text{s.t.}\quad & V_0 = cx_0 \leq cx^{'}, x^{'} \in all feasible
\end{align*}

将列生成算法应用于逆UFL问题,将上式对应的设施和运输部分写开可以得到该逆问题UFL的主问题

\begin{align*}
&\min \quad \sum_{i=1}^m(a_i+b_i)+\sum_{i=1}^m\sum_{j=1}^n(c_{ij}+d_{ij})\\
\text{s.t.}\quad & (a_i-b_i+f_i^0)v_i^{'} + (c_{ij}-d_{ij}+r_{ij}^0)u_{ij}^{'} - cx_0 \geq 0
\end{align*}

其中$(v_i^{'},u_{ij}^{'})$属于一个限制集,集合中的元素由子问题不断生成。
主问题得到的$(a^*,b^*,c^*,d^*)$带入到子问题中求解。

子问题如下:

\begin{align}
&\min \quad \sum_{i=1}^m(a_i^*-b_i^*+f_i^0)v_i^{'}+\sum_{i=1}^m\sum_{j=1}^n(c_{ij}^*-d_{ij}^*+r_{ij}^0)u_{ij}^{'}-cx_0\\
\text{s.t.}\quad & \sum_{i=1}^m u_{ij}^{'} =1 \quad \forall j\in L \\
& u_{ij}^{'} \leq v_{i}^{'} \quad \forall i \in F, \forall j \in L \\
& u_{ij}^{'} \geq 0 \quad \forall i \in F, \forall j \in L \\
&u_{ij}^{'} \in \{0,1\} ,\quad v_{i}^{'} \in \{0,1\}
\end{align}

子问题求出得到$(v^{'*},u^{'*})$, 如果目标函数值小于0,将得到的$(v^{'*},u^{'*})$加入到主问题的限制集中。如果目标函数值大于0,程序结束。

注意到子问题实际上是UFL原问题,由于UFL问题是NP-难问题,因此UFL逆问题也是NP-难问题。

可以对子问题进行如下的处理方法:\\
1. 精确求解子问题UFL,则可以得到UFL逆问题的最优解。\\
2. 近似求解子问题UFL,设定循环次数,提前结束子问题的求解可以得到UFL逆问题的下界。\\
3. 将子问题松弛得到分数解,则可以得到UFL逆问题的上界。




\subsection{减少约束得到下界}

注意这部分得到的gap较大

\subsection{启发式得到上界}

在2010文章中,定义了关于设施选址问题的

\[
\mathbb{C}_{\mathrm{UFL}}^{x}=\left\{(v, u) \in \mathbb{R}^{m+m v}: \sum_{i \in M} u_{i k}=1, \forall k \in V, v_{i}-u_{i k} \geq 0, \forall i \in M, \forall k \in V, u_{i k} \geq 0, \forall i \in M, \forall k \in V\right\}
\]


\begin{equation}
\begin{aligned}
&\min \sum_{i \in M} \left(\tau_{i}^{f}+\eta_{i}^{f}\right)+\sum_{i \in M} \sum_{k \in V} \left(\tau_{i k}^r+\eta_{i k}^{r}\right) \\
\text{s.t.} \quad &\sum_{k \in V} \pi_{k} \geq \nu_{\mathrm{UFL}}, \\
&\sum_{k \in V} \varrho_{i k}=\bar{f}_{i}, \forall i \in M, \\
&\pi_{k}-\varrho_{i k}+\zeta_{i k}=\bar{r}_{i k}, \quad \forall i \in M, \forall k \in V, \\
&\sum_{i \in M} \bar{f}_{i} v_{i}^{0}+\sum_{i \in M} \sum_{k \in V} \bar{r}_{i k} u_{i k}^{0}=\nu_{\mathrm{UFL}}, \\
& \tau_{i}^{f}-\eta_{i}^{f}=\bar{f}_{i}-f_{i}, \forall i \in M \quad \tau_{i k}^{r}-\eta_{i k}^{r}=\bar{r}_{i k}-r_{i k}, \forall i \in M, \forall k \in V, \\
&\nu_{\mathrm{UFL} 1} \leq \nu_{\mathrm{UFL}} \leq \nu_{\mathrm{UFL} 2}, \varrho_{i k} \geq 0, \zeta_{i k} \geq 0, \forall i \in M, \forall k \in V. \\
\end{aligned}
\end{equation}

其中$\bar{f}_{i}$和$\bar{r}_{ik}$ 为改变后的开启设施成本和运输成本。相对应的,$f_{i}$和$r_{ik}$ 为改变前的成本。
$\pi_{k},\varrho_{i k},\zeta_{i k}$分别为原UFL约束
\begin{equation*}
\sum_{i \in M} u_{ik} = 1,
v_i - u_{ik} \geq 0,
u_{ik} \geq 0
\end{equation*}
对应的对偶变量。 \par
$v_{i}^{0},u_{i k}^{0}$ 则对应于给定的可行解使之成为改变成本后的最优解。
$\nu_{\mathrm{UFL}}$ 是原UFL的最优值,需要在一定的范围内取值使得问题可行。

$\tau_{i}^{f},\eta_{i}^{f},\tau_{i k}^r,\eta_{i k}^{r}$ 则对应于求解范数一下引入的变量。 \par


这部分gap很小,实际上对应到(integrality gap)

\section{求解low bound 的一些方法}

\subsection{利用邻域的想法}

有如下约束限制可以帮助给定的可行解在一定范围内是最优的 \\
约束一(横向移动):对于给定的可行解(0-1)  对应的 $r_{ik}=1$ 需要小于其他所有 $r_{ik} = 0$ \\
约束二(0-1纵向移动):f 中为0 的部分 需要确保  $f_i +\min_i r_{ij} \geq \min r_{ik}$ \\
约束三(纵向移动):对于仅有一个 $r_ik=1$ 的行,考虑$r_{ik}+f_i \leq \min_j r_{jk}$ \\
约束四:考虑$f_i = 1$中 $r_{ik} =1$ 个数大于1的行 则有$\max f_i+ \sum r_{ik} \leq \min \text{对应列的} r_{ik}$

\section{有容量限制}

\subsection{列生成算法得到最优解}

\subsection{启发式得到上界}
对于有容量限制的设施选址逆优化问题,多了一个约束
\begin{eqation}
  \sum_k d_k u_{ik} \leq k_i v_i,i\in M
\end{eqation}。
同样利用锥优化的方法,添加对应上述约束的对偶变量$t$,我们可以得到有容量限制的设施选址问题的启发式算法

\begin{equation}
\begin{aligned}
&\min \sum_{i \in M} \left(\tau_{i}^{f}+\eta_{i}^{f}\right)+\sum_{i \in M} \sum_{k \in V} \left(\tau_{i k}^r+\eta_{i k}^{r}\right) \\
\text{s.t.} \quad &\sum_{k \in V} \pi_{k} \geq \nu_{\mathrm{UFL}}, \\
&k_it_i+\sum_{k \in V} \varrho_{i k}=\bar{f}_{i}, \forall i \in M, \\
&\pi_{k}-\varrho_{i k}+\zeta_{i k}-d_kt_i=\bar{r}_{i k}, \quad \forall i \in M, \forall k \in V, \\
&\sum_{i \in M} \bar{f}_{i} v_{i}^{0}+\sum_{i \in M} \sum_{k \in V} \bar{r}_{i k} u_{i k}^{0}=\nu_{\mathrm{UFL}}, \\
& \tau_{i}^{f}-\eta_{i}^{f}=\bar{f}_{i}-f_{i}, \forall i \in M \quad \tau_{i k}^{r}-\eta_{i k}^{r}=\bar{r}_{i k}-r_{i k}, \forall i \in M, \forall k \in V, \\
&\nu_{\mathrm{UFL} 1} \leq \nu_{\mathrm{UFL}} \leq \nu_{\mathrm{UFL} 2}, \varrho_{i k} \geq 0, \zeta_{i k} \geq 0, \forall i \in M, \forall k \in V. \\
\end{aligned}
\end{equation}






注意这部分 可以通过改变 需求和供给 的大小来退化得到 无容量限制


\end{document}
