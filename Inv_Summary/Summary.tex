\documentclass[UTF8]{article}
\author {Dis\cdot count}

\title {设施选址逆优化问题的计算}
\date{}
\usepackage{ctex}
\usepackage{amsmath}

\usepackage{geometry}
\geometry{a4paper,scale=0.8}
\usepackage{graphicx}
\usepackage{amssymb}

\usepackage{setspace}
\renewcommand{\baselinestretch}{1.5}


\usepackage{float}
\usepackage{color}%,soul}f
\usepackage{multirow}
\usepackage{xr}


\begin{document}
    \maketitle

\begin{abstract}

逆优化问题

我们使用了经典的列生成算法对UFL的逆优化问题进行了计算,其子问题就是求解原问题UFL,这样求解得到的就是UFL的逆优化最优值。当对子问题进行优化时,例如对子问题进行松弛得到非整数解这样可以得到与启发式方法一样的上界,而对子问题得到的非整数解取临近整数可以得逆问题的下界


\end{abstract}

\qquad \textbf{关键词: 逆优化、列生成算法、启发式算法 、设施选址问题}

\section{逆优化综述}  % Introduction

逆优化问题最初
已经有很多人进行了研究
同时设施选址

\section{设施选址逆优化}

\section{无容量限制}




\subsection{列生成算法得到最优解}



\subsection{减少约束得到下界}

注意这部分得到的gap较大

\subsection{启发式得到上界}

在2010文章中,定义了关于设施选址问题的


这部分gap很小

考虑在原问题上改进得到更好上界

\section{有容量限制}

\subsection{列生成算法得到最优解}

\subsection{减少约束得到下界}


\subsection{启发式得到上界}

注意这部分 可以通过改变 需求和供给 的大小来退化得到 无容量限制


\end{document}
