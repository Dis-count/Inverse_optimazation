\documentclass[UTF8]{article}

\usepackage{graphicx}
\usepackage{setspace}
% \usepackage{ctex}
\usepackage{amsmath}
\usepackage{geometry}


\newtheorem{thm}{Theorem}

\newtheorem{pf}{Proof}
\newtheorem{algorithm}{Algorithm}


% \geometry{a4paper,left=2cm,right=2cm,top=1cm,bottom=1cm}
% \setstretch{1.5}
\geometry{a4paper,scale=0.8}
\renewcommand{\baselinestretch}{1.5}

\title{The literature review of the nine papers}
\author{Dis\cdot count}
% \date{Feb 2019}
\begin{document}

\maketitle{}

\section{Summary}

There are

\section{Some reverse location problems(2000)}

This article discusses about the problem of facilities locations with the fixed locations, and the optimal solution is also not the same.

So it is called \emph{reverse} problem. For the tree network problem, use the minimum cut or maximum flow algorithm (strongly polynomial method) as main subroutine.


\section{Solving Inverse Spanning Tree Problems Through Network Flow Techniques(1999)}

They first study the inverse spanning tree problems which can be transformed to an assignment problem. So just solve the unbalaned assignment problem


\section{Inverse Combinatorial Optimization: A Survey on Problems, Methods, and Results(2004)}

This overview should be read carefully.



\section{Inverse Optimization(2001)}

This article shows many cases.



\section{Inverse Polynomial Optimization(2013)}




This article provides a systematic numerical scheme to compute an inverse optimal solution.



\section{The inverse optimal value problem(2005)}

Thispaperconsidersthefollowinginverseoptimizationproblem:givenalinearprogram,adesired optimal objective value, and a set of feasible cost vectors, determine a cost vector such that the corresponding optimal objective value of the linear program is closest to the desired value.
The above problem, referred here as the inverse optimal value problem, is significantly different from standard inverse optimization problems thatinvolvedeterminingacostvectorforalinearprogramsuchthatapre-specifiedsolutionvectorisoptimal. In this paper, we show that the inverse optimal value problem is NP-hard in general. We identify conditions underwhichtheproblemreducestoaconcavemaximizationoraconcaveminimizationproblem.Weprovide sufficient conditions under which the associated concave minimization problem and, correspondingly, the inverse optimal value problem is polynomially solvable. For the case when the set of feasible cost vectors is polyhedral, we describe an algorithm for the inverse optimal value problem based on solving linear and bilinear programming problems. Some preliminary computational experience is reported.


\section{Inverse integer programming(2009)}

Theoretical

We consider the integer programming version of inverse optimization. Using superadditive duality, we provide a polyhedral description of the set of inverse feasible objectives. We then describe two algorithmic approaches for solving the inverse integer programming problem.

We consider inverse integer programming, where an integer vector x0, is given, as well as a constraint matrix, right-handside and a target objective.The goal is to find a vector d that minimizes the weighted norm from a target objective d0 such that x0 is optimal for the pure integer program defined by the objective d. Algorithms for inverse linear programming have been developed and refined in [1] and [8]. The inverse counterparts of various combinatorial optimization problems have been described, including shortest paths, spanning trees, and minimum cost flows. Ahuja and Orlin [1] showed that, under mild conditions, the inverse version of a polynomially solvable optimization problem under the L1 and L∞ norms are polynomially solvable. Less is known aboutinverseintegerprogramming.Huang[6]showedthattheinverseknapsackproblemandthegeneralinverseintegerprogrammingproblemwithafixednumberofrows can be solved in pseudo-polynomial time. See the recent extensive survey of inverse combinatorial optimization by Heuberger [5] for more details.



\section{Cutting plane algorithms for the inverse mixed integer linear programming problem(2009)}

Theoretical


\section{Calculating some inverse linear programming problems(1996)}

A method for solving general inverse LP problem including upper and lower bound constraints is suggested which is based on the optimality conditions for LP problems. It is found that when the method is applied to \emph{inverse minimum cost flow problem} or \emph{inverse assignment problem}, we are able to obtain strongly polynomial algorithms.


In [2], Burton and Toint considered the computations of an inverse shortest path problem. As they use L2 norm to measure closeness between two vectors, they transform the inverse problem into a quadratic programming problem to solve.

In [7], Zhang et al. use L1 norm and deal with the inverse shortest path problem as a special LP problem.

The combinatorial structure of the feasible region for inverse shortest path problem is exposed in [5].

Some inverse minimum spanning tree problems are studied in [4, 6].

The method is based on the optimality conditions for LP problems. As some particular examples, we shall apply the method to two classes of problems: inverse minimum cost flow problem and inverse assignment problem.

It will be seen that when this method is used, the calculation of the first problem is reduced to a minimum cost circulation problem, whereas the second one can be solved by calculating the original assignment problem or a minimum cost circulation problem depending on different additional requirements.

In all these cases we are able to obtain strongly polynomial algorithms. This paper is organized as follows.

In Section 2, we first prove equivalence of the general inverse LP problem to other two related models, from which a method for calculating inverse LP problems is derived.

Section 3 is devoted to the application of the method to solve inverse minimum cost flow problems.

In Section 4 we use the method to calculate inverse assignment problems. If some weights in the assignment model are not allowed to adjust, we call such problems restricted inverse assignment problems, and a special case of which is solved in Section 5.



\end{document}
